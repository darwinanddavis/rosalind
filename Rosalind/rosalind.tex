\documentclass[10,portrait]{article}
\usepackage{lmodern}
\usepackage{amssymb,amsmath}
\usepackage{ifxetex,ifluatex}
\usepackage{fixltx2e} % provides \textsubscript
\ifnum 0\ifxetex 1\fi\ifluatex 1\fi=0 % if pdftex
  \usepackage[T1]{fontenc}
  \usepackage[utf8]{inputenc}
\else % if luatex or xelatex
  \ifxetex
    \usepackage{mathspec}
  \else
    \usepackage{fontspec}
  \fi
  \defaultfontfeatures{Ligatures=TeX,Scale=MatchLowercase}
\fi
% use upquote if available, for straight quotes in verbatim environments
\IfFileExists{upquote.sty}{\usepackage{upquote}}{}
% use microtype if available
\IfFileExists{microtype.sty}{%
\usepackage[]{microtype}
\UseMicrotypeSet[protrusion]{basicmath} % disable protrusion for tt fonts
}{}
\PassOptionsToPackage{hyphens}{url} % url is loaded by hyperref
\usepackage[unicode=true]{hyperref}
\PassOptionsToPackage{usenames,dvipsnames}{color} % color is loaded by hyperref
\hypersetup{
            pdftitle={Rosalind},
            pdfauthor={1 Department of Biology, Emory University, 1510 Clifton Road NE, Atlanta, GA, USA, 30322},
            colorlinks=true,
            linkcolor=blue,
            citecolor=red,
            urlcolor=blue,
            breaklinks=true}
\urlstyle{same}  % don't use monospace font for urls
\usepackage[margin=1in]{geometry}
\usepackage[]{biblatex}
\usepackage{color}
\usepackage{fancyvrb}
\newcommand{\VerbBar}{|}
\newcommand{\VERB}{\Verb[commandchars=\\\{\}]}
\DefineVerbatimEnvironment{Highlighting}{Verbatim}{commandchars=\\\{\}}
% Add ',fontsize=\small' for more characters per line
\usepackage{framed}
\definecolor{shadecolor}{RGB}{248,248,248}
\newenvironment{Shaded}{\begin{snugshade}}{\end{snugshade}}
\newcommand{\KeywordTok}[1]{\textcolor[rgb]{0.13,0.29,0.53}{\textbf{#1}}}
\newcommand{\DataTypeTok}[1]{\textcolor[rgb]{0.13,0.29,0.53}{#1}}
\newcommand{\DecValTok}[1]{\textcolor[rgb]{0.00,0.00,0.81}{#1}}
\newcommand{\BaseNTok}[1]{\textcolor[rgb]{0.00,0.00,0.81}{#1}}
\newcommand{\FloatTok}[1]{\textcolor[rgb]{0.00,0.00,0.81}{#1}}
\newcommand{\ConstantTok}[1]{\textcolor[rgb]{0.00,0.00,0.00}{#1}}
\newcommand{\CharTok}[1]{\textcolor[rgb]{0.31,0.60,0.02}{#1}}
\newcommand{\SpecialCharTok}[1]{\textcolor[rgb]{0.00,0.00,0.00}{#1}}
\newcommand{\StringTok}[1]{\textcolor[rgb]{0.31,0.60,0.02}{#1}}
\newcommand{\VerbatimStringTok}[1]{\textcolor[rgb]{0.31,0.60,0.02}{#1}}
\newcommand{\SpecialStringTok}[1]{\textcolor[rgb]{0.31,0.60,0.02}{#1}}
\newcommand{\ImportTok}[1]{#1}
\newcommand{\CommentTok}[1]{\textcolor[rgb]{0.56,0.35,0.01}{\textit{#1}}}
\newcommand{\DocumentationTok}[1]{\textcolor[rgb]{0.56,0.35,0.01}{\textbf{\textit{#1}}}}
\newcommand{\AnnotationTok}[1]{\textcolor[rgb]{0.56,0.35,0.01}{\textbf{\textit{#1}}}}
\newcommand{\CommentVarTok}[1]{\textcolor[rgb]{0.56,0.35,0.01}{\textbf{\textit{#1}}}}
\newcommand{\OtherTok}[1]{\textcolor[rgb]{0.56,0.35,0.01}{#1}}
\newcommand{\FunctionTok}[1]{\textcolor[rgb]{0.00,0.00,0.00}{#1}}
\newcommand{\VariableTok}[1]{\textcolor[rgb]{0.00,0.00,0.00}{#1}}
\newcommand{\ControlFlowTok}[1]{\textcolor[rgb]{0.13,0.29,0.53}{\textbf{#1}}}
\newcommand{\OperatorTok}[1]{\textcolor[rgb]{0.81,0.36,0.00}{\textbf{#1}}}
\newcommand{\BuiltInTok}[1]{#1}
\newcommand{\ExtensionTok}[1]{#1}
\newcommand{\PreprocessorTok}[1]{\textcolor[rgb]{0.56,0.35,0.01}{\textit{#1}}}
\newcommand{\AttributeTok}[1]{\textcolor[rgb]{0.77,0.63,0.00}{#1}}
\newcommand{\RegionMarkerTok}[1]{#1}
\newcommand{\InformationTok}[1]{\textcolor[rgb]{0.56,0.35,0.01}{\textbf{\textit{#1}}}}
\newcommand{\WarningTok}[1]{\textcolor[rgb]{0.56,0.35,0.01}{\textbf{\textit{#1}}}}
\newcommand{\AlertTok}[1]{\textcolor[rgb]{0.94,0.16,0.16}{#1}}
\newcommand{\ErrorTok}[1]{\textcolor[rgb]{0.64,0.00,0.00}{\textbf{#1}}}
\newcommand{\NormalTok}[1]{#1}
\IfFileExists{parskip.sty}{%
\usepackage{parskip}
}{% else
\setlength{\parindent}{0pt}
\setlength{\parskip}{6pt plus 2pt minus 1pt}
}
\setlength{\emergencystretch}{3em}  % prevent overfull lines
\providecommand{\tightlist}{%
  \setlength{\itemsep}{0pt}\setlength{\parskip}{0pt}}
\setcounter{secnumdepth}{0}
% Redefines (sub)paragraphs to behave more like sections
\ifx\paragraph\undefined\else
\let\oldparagraph\paragraph
\renewcommand{\paragraph}[1]{\oldparagraph{#1}\mbox{}}
\fi
\ifx\subparagraph\undefined\else
\let\oldsubparagraph\subparagraph
\renewcommand{\subparagraph}[1]{\oldsubparagraph{#1}\mbox{}}
\fi

% set default figure placement to htbp
\makeatletter
\def\fps@figure{htbp}
\makeatother


\title{Rosalind}
\author{\emph{\textsuperscript{1} Department of Biology, Emory University, 1510
Clifton Road NE, Atlanta, GA, USA, 30322} \footnote{`This Supplementary
  Material can be found at
  \url{https://github.com/darwinanddavis/rosalind}'}}
\date{}

\begin{document}
\maketitle

{
\hypersetup{linkcolor=black}
\setcounter{tocdepth}{3}
\tableofcontents
}
~

Date: 2018-08-10\\
R version: 3.5.0\\
*Corresponding author:
\href{mailto:matthew.malishev@gmail.com}{\nolinkurl{matthew.malishev@gmail.com}}

\newpage   

\subsection{Overview}\label{overview}

This document contains the code and solutions to the coding problem sets
from the Rosalind database at
\url{http://rosalind.info/problems/list-view/}

\begin{center}\rule{0.5\linewidth}{\linethickness}\end{center}

~

Install packages and set working dir

\begin{Shaded}
\begin{Highlighting}[]
\CommentTok{# install pcks}
\NormalTok{packages <-}\StringTok{ }\KeywordTok{c}\NormalTok{(}\StringTok{"stringr"}\NormalTok{,}\StringTok{"stringi"}\NormalTok{,}\StringTok{"gsubfn"}\NormalTok{)   }
\ControlFlowTok{if}\NormalTok{ (}\KeywordTok{require}\NormalTok{(packages)) \{}
  \KeywordTok{install.packages}\NormalTok{(packages,}\DataTypeTok{dependencies =}\NormalTok{ T)}
  \KeywordTok{require}\NormalTok{(packages)}
\NormalTok{\}}
\KeywordTok{lapply}\NormalTok{(packages,library,}\DataTypeTok{character.only=}\NormalTok{T)}
\end{Highlighting}
\end{Shaded}

\subsubsection{}\label{section}

\subsection{Problems}\label{problems}

\subsubsection{DNA Counting DNA
Nucleotides}\label{dna-counting-dna-nucleotides}

\begin{Shaded}
\begin{Highlighting}[]
\NormalTok{s <-}\StringTok{ }\KeywordTok{read.csv}\NormalTok{(}\StringTok{"rosalind_dna.txt"}\NormalTok{,}\DataTypeTok{header=}\NormalTok{F,}\DataTypeTok{sep=}\StringTok{","}\NormalTok{,}\DataTypeTok{stringsAsFactors =}\NormalTok{ F);s}
\end{Highlighting}
\end{Shaded}

\begin{verbatim}
## # A tibble: 1 x 1
##   V1                                                                      
##   <chr>                                                                   
## 1 AAGTGCTAGATCTGCGGGCCAACTCTCTATGACCCGAGACAGCCGCGGCTCTCTTACACCGTAGGCATGAT~
\end{verbatim}

\begin{Shaded}
\begin{Highlighting}[]
\KeywordTok{str_count}\NormalTok{(s,}\KeywordTok{c}\NormalTok{(}\StringTok{"A"}\NormalTok{,}\StringTok{"C"}\NormalTok{,}\StringTok{"G"}\NormalTok{,}\StringTok{"T"}\NormalTok{))}
\end{Highlighting}
\end{Shaded}

\begin{verbatim}
## [1] 199 224 210 205
\end{verbatim}

\subsubsection{RNA Transcribing DNA into
RNA}\label{rna-transcribing-dna-into-rna}

\begin{Shaded}
\begin{Highlighting}[]
\NormalTok{t <-}\StringTok{ }\KeywordTok{read.csv}\NormalTok{(}\StringTok{"rosalind_rna.txt"}\NormalTok{,}\DataTypeTok{header=}\NormalTok{F,}\DataTypeTok{sep=}\StringTok{","}\NormalTok{,}\DataTypeTok{stringsAsFactors =}\NormalTok{ F);t}
\end{Highlighting}
\end{Shaded}

\begin{verbatim}
## # A tibble: 1 x 1
##   V1                                                                      
##   <chr>                                                                   
## 1 GAACGAGGAACGTCTTGCCACCATCCGATCTGAAACGGCAGGTACGTTTACAAAGTTCTCCAGTGTAAAAC~
\end{verbatim}

\begin{Shaded}
\begin{Highlighting}[]
\KeywordTok{gsub}\NormalTok{(}\StringTok{"T"}\NormalTok{,}\StringTok{"U"}\NormalTok{,t)}
\end{Highlighting}
\end{Shaded}

\begin{verbatim}
## [1] "GAACGAGGAACGUCUUGCCACCAUCCGAUCUGAAACGGCAGGUACGUUUACAAAGUUCUCCAGUGUAAAACACGGUGCGGUGAAGAGUUCCCAAUCCAGGGAGCCUUCAAAGAGACCAAUUUGGCUGUUCGGGCUUCGAUCCACGCGAUCUUGGAAUGCAGUCACGCCAUAGGAUAGCCGACACUUUGGAUGGAGCCUAAAGUCCAAGACGCCGUUUUACCGGCUCAUUAGGUACGUGACUAUAGACGCUACGCGGUCGGUUCCAUAGUUCCUGUCCUGGCAUCGAGGGGUGUCGACCAGUAAAUCCGUUUGCUCACCACCAAGGUCUUCGUAUAUCCAUGCGCGUGAACGGCGGUGAAAGUGAAACCAUAGGGCGCCCGGCGGGCAACGUAGCGUUCAUACCUAUUCCUGCGUCUCUGAAAAUCGCGACAUAGGUAGGCCCCACGUACGGUAUAUUUCCUAGCCUACGCCUGUUUAUACCCGCGCGGAUUGGGGGAAGAAUAUGUGUUCUGAAAAAGAGUAAAAUGUUGGGCUACGGUCGAGCCUAAAGACAGCAGGCCGCUCCCUAUCCUCUGGCCGACGUACGUUUGUUGUCUAUUGGUGAAACUCAUUCAAGAUUGCGCAUUCAAUCACACAGUAUGUUAUGUAUUAGAAAAGAUCUGCGUAUAUCCGCUCAACGUUACAACUGCAUUUCAAGAUGUACUACAUAGUGGAUGCACUGAACCAUAGCGGCCGUCCUAGCUAAGGAAAUGAUAAUACCCGCUACGGACUUCAUUACAAGACUGUGCAUCAGCAGUUACUAGGUAGCUUUUUCGCGGUUCCAAUAAACAUUGCGAUGGCGACAGAGUCUGCACAAAAAGCUAACGACUGCACAUUUGUCUGAGCAAACGGAUUAUAGGUAACAGCAUGUAGCGGCGCUACAAACCGGUGCGUUCUAAG"
\end{verbatim}

\subsubsection{Complementing a Strand of
DNA}\label{complementing-a-strand-of-dna}

\begin{Shaded}
\begin{Highlighting}[]
\KeywordTok{require}\NormalTok{(stringi)}
\KeywordTok{require}\NormalTok{(gsubfn)}

\CommentTok{# s <- read.csv(".txt",header=F,sep=",",stringsAsFactors = F);s}

\NormalTok{s <-}\StringTok{ "AAAACCCGGT"}
\NormalTok{sc <-}\StringTok{ }\KeywordTok{c}\NormalTok{(}\StringTok{"A"}\NormalTok{,}\StringTok{"T"}\NormalTok{,}\StringTok{"C"}\NormalTok{,}\StringTok{"G"}\NormalTok{) }\CommentTok{# target units}
\NormalTok{sc2r <-}\StringTok{ }\KeywordTok{c}\NormalTok{(}\StringTok{"T"}\NormalTok{,}\StringTok{"A"}\NormalTok{,}\StringTok{"G"}\NormalTok{,}\StringTok{"C"}\NormalTok{) }\CommentTok{# units to be replaced}
\NormalTok{s <-}\StringTok{ }\NormalTok{stringi}\OperatorTok{::}\KeywordTok{stri_reverse}\NormalTok{(s);s }\CommentTok{# reverse string}
\end{Highlighting}
\end{Shaded}

\begin{verbatim}
## [1] "TGGCCCAAAA"
\end{verbatim}

\begin{Shaded}
\begin{Highlighting}[]
\NormalTok{s <-}\StringTok{ }\KeywordTok{replace}\NormalTok{(s,sc,sc2r);s }\CommentTok{# replace }
\end{Highlighting}
\end{Shaded}

\begin{verbatim}
##                         A            T            C            G 
## "TGGCCCAAAA"          "T"          "A"          "G"          "C"
\end{verbatim}

\begin{Shaded}
\begin{Highlighting}[]
\CommentTok{# desired result: ACCGGGTTTT}
\end{Highlighting}
\end{Shaded}

\printbibliography

\end{document}
